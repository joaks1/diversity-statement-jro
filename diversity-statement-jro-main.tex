\section*{My Philosophy on Diversity in Science}
I am passionate about exploring open-ended questions about the natural world
that require critical thinking and creativity.
% Such endeavors benefit greatly from the insights of people with diverse
% backgrounds and perspectives.
Such endeavors benefit greatly from the insights of people from across
many dimensions of diversity,
including socioeconomic status, ethnicity, race, culture, gender, and
sexuality.
To create such a diverse research community, we must first educate a diverse
population of students about science.
To do this, we must
create a welcoming environment,
make science relevant to students from diverse backgrounds,
and
% present concepts from multiple perspectives to accommodate diverse styles of
% learning.
use teaching methods that are supported by evidence to improve learning and
close historical achievement gaps.
A diverse academic community serves to broaden perspectives, bring together
unique insights, and inspire creativity in our pursuit to understand the
natural world and impart what we find to our students.

\section*{My Experiences with Diversity in STEM Education}
As an undergraduate, I was a tutor for Student Support Services at the
University of Wisconsin Oshkosh.
This allowed me to help first-generation and special-needs students understand
key concepts in biology and chemistry via one-on-one consultations.
Many of my students were socioeconomically underprivileged and/or members of
ethnic or racial minority groups.
By working with students in this setting, I quickly learned to present ideas
from a variety of perspectives to accommodate diverse learning styles.

As a first-generation college graduate, I am driven
% to do whatever I can
to make material relevant to students with little or no science background.
I was lucky enough to get involved in research projects as an undergraduate
student under the mentorship of multiple biology faculty members.
As a young student naive to the pursuit of science, actively engaging in
research was critical to my education and career.
Due to these experiences, I am passionate about getting underprivileged
students involved in research.

I am often involved in international workshops that teach undergraduate and
graduate students, postdocs, and faculty about computational biology and using
scientific software.
English is not the first language of many of the attendees.
From these workshops, I learned to make difficult concepts as relevant and
discernible as possible by designing
% and implementing
inquiry-based exercises and tutorials.
Similarly, when I taught Introductory Biology at the University of
Washington, I was accompanied at the front of the class by a sign-language
interpreter who translated my lectures for my hearing-impaired students.
This made me acutely aware of the diversity of challenges faced by
students, and the importance of being as clear and precise as possible when
presenting material.
Furthermore, I have found that active-learning exercises help level the
classroom and allow students to excel that would otherwise struggle in a
traditional lecture setting.
From my work with the UW
\href{https://sites.google.com/site/uwbioedresgroup/home}{Biology Education
    Research Group}, I have learned to teach by asking, rather than telling.

% My empirical research is geographically focused in Central and Southeast Asia.
% This gives me the opportunity to mentor international undergraduate and
% graduate students from institutions such as the Universiti Kebangsaan Malaysia
% and Universiti Sains Malaysia in the methods of field-based biodiversity
% research.
% This work allows me to experience firsthand the benefits when people from
% diverse backgrounds work together.
% For example, I have gained invaluable insights from these students about
% natural history and the anthropological pressures on the biodiversity in these
% regions.
% I plan to maximize this mutual benefit and leverage my international
% collaborations to recruit motivated students from Malaysia, Thailand, and
% Mongolia to join my lab.

\subsubsection*{My Experiences at Auburn University (AU)}
While at AU, I have worked to maintain an even ratio of female to male
undergraduate and graduate students in my lab.
However, being in the deep south, I have found it challenging to recruit
diverse students and postdocs to AU.
Several minority graduate students in my department have told me that they are
not comfortable on campus or living in Auburn, which is conspicuously
nondiverse.
Despite being just north of the Black Belt,
where African Americans comprise over 53\% of the population,
African Americans make up
16\% of Auburn's population, less than 7\% of the student body,
and only 4\% of faculty.
Nonetheless, I worked with
% several great people in
% our Office of Inclusion, Equity, and Diversity (OIED) to successfully recruit a
our Office of Inclusion, Equity, and Diversity to successfully recruit a
talented Native Tibetan Nomad as a Ph.D.\ student
and an African American undergraduate researcher,
and I continue to work with their
\href{https://www.auburn.edu/cosam/departments/diversity/summerbridge/index.htm}{STEM Summer Bridge Program}
to help recruit and prepare students from traditionally underrepresented groups
to be successful at AU.
% The challenges the OIED and I encountered in recruiting my Tibetan student made
% it clear that the higher administration at AU makes diversity a low priority.
% One does not have to look hard to understand why.
% These numbers are a reflection of a deeply rooted culture at this institution.
% Last year, AU scheduled, canceled, then ultimately allowed
% neo-Nazi Richard Spencer to speak on campus.
% I do not mean to be overly disparaging of my institution.
% One example that illustrates this problem is my recent recruitment of a
% talented Ph.D.\ student who comes from underprivileged circumstances in Tibet.
% After receiving help from colleagues in our diversity office to successfully
% recruit her, I was met with challenges from all levels of AU administration
% when I sought approval to use start-up funds to defray the cost of her move to
% Auburn,
% which after eight months ultimately resulted in the use of personal funds to
% get her here.
% There was no maliciousness on the part of AU or its employees,
% but it was clear that this institution does not value the recruitment of
% talented people from diverse backgrounds.


\subsubsection*{Prison Education}
% Given the challenging situation at AU,
Given the diversity challenge at AU,
I have sought out other ways to bring science education to diverse communities.
Arguably, incarcerated adults are the most underserved population in this country,
yet society has much to gain from prisoners receiving a quality
education\footnote{\label{Vacca04}\shortfullcite{Vacca2004}}.
The Alabama Prison + Arts Education Project (APAEP) was founded in 2003 by Kyes
Stevens with the goal of providing quality
educational experiences to the prison population of Alabama.
The classes provided by APAEP have been predominantly focused on the
arts and humanities.
Since arriving at AU, I have been working with the APAEP to develop three
14-week courses in evolutionary biology, with the long-term goal of
incorporating a broad set of STEM courses into the APAEP curriculum.
I am currently seeking funding in the outreach components of my NSF proposals
to incorporate computer science classes.
Teaching for the APAEP has been my most rewarding experience as an educator.
My students within Alabama correctional facilities are exceptionally
hard-working and eager to learn, and I am privileged to get to work with them.
