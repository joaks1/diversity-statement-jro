I am passionate about exploring open-ended questions about the natural world
that require critical thinking and creativity.
Such endeavors benefit greatly from the insights of people with diverse
backgrounds and perspectives.
To create such a diverse research community, we must first educate a diverse
population of students about science.
To do this effectively, we must be able to make science relevant to students
with no scientific background, and present concepts from multiple perspectives
to accommodate diverse styles of learning.
A racially, ethnically, politically, and socioeconomically diverse academic
community serves to broaden perspectives, bring together unique insights, and
inspire creativity in our pursuit to understand the natural world and impart
what we find to our students.

As an undergraduate, I was a tutor for Student Support Services at the
University of Wisconsin Oshkosh.
This allowed me to help first-generation and special-needs students understand
key concepts in biology and chemistry via one-on-one consultations.
Many of my students were socioeconomically underprivileged or members of ethnic
or racial minority groups.
By working with students in this setting, I quickly learned to present ideas
from a variety of perspectives to accommodate diverse learning styles.

As a first-generation college graduate, I am driven to do whatever I can to
make material relevant to students with little or no science background.
I was lucky enough to get involved in research projects as an undergraduate
student under the mentorship of multiple biology faculty members.
As a young student naive to the pursuit of science, actively engaging in
research was critical to my education and career.
Due to these experiences, I am passionate about getting underprivileged
students involved in research.

I am often involved in international workshops that teach undergraduate and
graduate students, postdocs, and faculty about computer programming and/or
using scientific software.
English is not the first language of many of the attendees.
From these workshops, I learned to make difficult concepts as relevant and
discernable as possible by designing and implementing inquiry-based exercises
and tutorials.
Similarly, when I teach Introductory Biology (BIOL 180) at the University of
Washington, I am accompanied at the front of the class by a sign-language
interpreter who translates my lectures for my hearing-impaired students.
This has made me acutely aware of the diversity of challenges faced by
students, and the importance of being as clear and precise as possible when
presenting material.
Furthermore, I have found that active-learning exercises help level the
classroom and allow students to excel that would otherwise struggle in a
traditional lecture setting; I have learned to teach by asking, rather than
telling.

My empirical research is geographically focused in Central and Southeast Asia.
This gives me the opportunity to mentor international undergraduate and
graduate students from institutions such as the Universiti Kebangsaan Malaysia
and Universiti Sains Malaysia in the methods of field-based biodiversity
research.
This work allows me to experience firsthand the benefits when people from
diverse backgrounds work together.
For example, I have gained invaluable insights from these students about
natural history and the anthropological pressures on the biodiversity in these
regions.
I plan to maximize this mutual benefit and leverage my international
collaborations to recruit motivated students from Malaysia, Thailand, and
Mongolia to join my lab.

