I am passionate about exploring questions about the natural world that are
open-ended and require critical thinking and creativity.
Such endeavors benefit greatly from the insights of people with diverse
backgrounds and perspectives.
To create such a diverse research community, we must first educate a diverse
population of students about science and research.
To do this effectively, we must make science relevant to students with no
scientific background and present concepts from multiple perspectives to
accommodate diverse styles of learning.
A racially, ethnically, politically, and socioeconomically diverse academic
community serves to broaden perspectives, bring together unique insights, and
inspire creativity in our pursuit to understand the natural world and impart
what we find to our students.

As an undergraduate, I was a tutor for Student Support Services at the
University of Wisconsin Oshkosh.
This allowed me to help first-generation and special-needs students understand
key concepts in biology and chemistry via one-on-one consultations.
Many of my students were socioeconomically underprivileged or members of ethnic
or racial minority groups.
By working with students in this setting, I quickly learned to
present ideas from a variety of perspectives to accommodate diverse learning
styles.

I am a first-generation college graduate, and
this inspires me to do whatever I can to make material relevant to students
with little or no science background.
As an undergraduate student, I was lucky enough to get involved in research
projects under the mentorship of multiple biology faculty members.
These experiences made me aware of the importance of research in undergraduate
science education, especially for students that have never been exposed to the
pursuit of science.
Understanding how critical my undergraduate research experience was to my
science education, I am passionate about getting underprivileged students
involved in my research.

I am often involved in international workshops that teach undergraduate and
graduate students, postdocs, and faculty about computer programming or how to
use scientific software.
English is not the first language of many of the attendees.
This has given me experience in making difficult concepts as relevant and
discernable as possible by designing and implementing inquiry-based exercises
and tutorials.
When I teach Introductory Biology (BIOL 180) at the University of Washington, I
am accompanied at the front of the class by a sign-language interpreter who
translates my lectures for my hearing-impaired students.
This has helped make me acutely aware of how I present material, and motivates
me to be as clear and precise as possible.


My empirical research is geographically focused in Central and Southeast Asia.
This gives me the opportunity to mentor undergraduate and graduate students
from institutions such as the Universiti Kebangsaan Malaysia and Universiti Sains
Malaysia in the methods of field-based biodiversity research.
This work is extremely rewarding and often is a two-way street.
For example, I have gained invaluable insights from these students about the
anthropological pressures on the biodiversity in these regions.
I plan to maximize this mutual gain and leverage my international
collaborations to recruit motivated international students to join my lab.

